\documentclass[12pt]{article}
\usepackage{mathtools}
\begin{document}

\title{$\Lambda$CDM and Modified Gravity: Descriptions of The Observed Universe at Different Cosmological Scales}
\author{Imran Hasan}
\date{December 2016}
\maketitle


Einstein''s theory of gravity relates the curvature of spacetime to the stress energy tensor.

$$G_{\mu\nu} = R_{\mu \nu} - \frac{1}{2} g_{\mu\ \nu} R = 8\pi GT_{\mu \nu}$$

The theory has withstood several observational tests. Among them are Gravity Probe B's measurement of gravitomagnitism, and Eddington's measurement of the deflection of light but he sun. Additionally, GR is able to explain the advancement of the perihelion in Mercury's orbit, and recovers Newtonian gravity in the appropriate limits. However, these tests occurred at length and mass scales no larger than those found in the solar system. On larger scales (galactic, cluster, and cosmological) there is tension between the Einstein equations as described above, and astronomical and cosmological measurements. Measurements in these mass and size regimes do not agree with Einstein's theory, as understood above. For example, rotation curves of spiral galaxies indicate their stars and gas have circular velocities that are too large to be explained by luminous matter alone. While we do not claim rotation curves to be the only source of tension between theory and observation, this particular example provides a good framework for the topic of this paper

%[figure about a rotation curve. look at https://arxiv.org/pdf/astro-ph/9909252v1.pdf for a rotation curve of M33. "The extended rotation curve and the dark matter halo of M33". Monthly Notices of the Royal Astronomical Society. 311 (2): 441?447. Look at http://articles.adsabs.harvard.edu//full/1991MNRAS.249..523B/0000527.000.html for a whole lot of them]

The rotation curves presented in the literature cannot be explained by Einstein's gravity and baryonic matter (stars, dust, gas and so on) alone. The best fit lines through the data, and the velocity profiles for the luminous matter disagree considerably. Most notable is the flattening feature of the data at large radii, which is inconsistent with the gas and luminous curves, which tapper off. Clearly, something is amiss. 

One way to resolve the tension is to add non-baryonic forms of matter-energy to the Stress Energy Tensor. By adding a \emph{dark matter halo}' to the density profile of spiral galaxies, we may recover the observed rotation curves. Another strategy is to alter the theory of gravity all together. The rotation curves may be recovered if we include correction terms to Newtonian gravity. One may view these perspectives as altering the right hand side of equation [1] for the former, and altering the left hand side of equation [1] for the latter. 

The previous example presents a boiler plate of how the remainder of the paper will follow. We will examine the abilities of dark matter/dark energy models and modified gravity models to explain observation, across different length and mass scales. To do this rigorously, we must first define dark matter and dark energy, and discuss some theories of modified gravity. In the following three sections we discuss dark matter/dark energy in the $\Lambda CDM$ cosmological mode, and non-relativistic and relativistic models of modified gravity. However, we restrict ourselves to brief descriptions of these theories and how they fit into the overall description of GR, so that we may discuss their strengths and weaknesses at length later in the paper.

\section{$\Lambda$CDM}

Although there are several dark matter/dark energy models to choose from, one of the most tested and widely successful comes from $\Lambda$CDM cosmology, sometimes referred to as "standard cosmology". This paradigm can be understood by starting with the Friedmann Walkerson metric and Friedmann equations,

$$ds^{2} = -dt^{2} + a^{2}(t) (\frac{dr^{2}}{1 - kr^{2}} + r^{2}d\Omega^{2})$$

$$ H^{2} + \frac{k}{a^2} = \frac{8\pi G_{N}}{3} \rho_{tot}$$

where H is the Hubble Parameter, $\frac{\dot{a(t)}}{a(t)}$ \cite{Bertone2004}. In $\Lambda$CDM, we take k from equation to be zero, corresponding to a flat universe. From the Friedmann equation, we can see this leads to a restriction on the Stress Energy Tensor, $\rho_{c} = \frac{3H^2}{8piG}$. We then describe$ \rho_{c}$ as being made up of different species,


$$\Omega_{i} = \frac{\rho_{i}}{\rho_{c}}$$

where 

$$\Omega = \sum_{i} \Omega_{i}$$

Where the different species are $\Omega_{M}$, $\Omega_{R}$, $\Omega_{\Lambda}$. $\Omega_{R}$, from radiation, is at a scale of ~$10^{-4}$, and thus the mass-energy budget is dominated by $\Omega_{M}$ and $\Omega_{\Lambda}$, matter and dark energy densities respectively \cite{Caroll2004}.

$\Omega_{M}$ has been measured by experiments like WMAP, SDSS, and PLANK. The Plank measurement has the smallest error bars, and is consistent with the aforementioned experiments, and gives $\Omega_{M}$ = .308 +\- 0.012 , which implies $\Omega_{\Lambda}$ to be approximately .7 \cite{Plank2015}, \cite{SDSS2003}%https://arxiv.org/abs/1502.01589 https://arxiv.org/abs/astro-ph/0310723

$\Omega_{M}$ can be further subdivided into two categories, baryonic matter (the usual matter we encounter day to day, made up of protons and neutrons) and non-baryonic matter. Measurements of anisotropies in the cosmic microwave background , measurements of quasar spectra, and measurements of luminous mass in galaxy clusters puts $\Omega_{b}$ to be within 2-5 percent of $\rho_{c}$. This means the remaining matter in the universe is nonbaryonic \cite{scott2002}.

Observational tests and numerical simulations have had some success in describing non-baryonic matter, or dark matter, as it is sometimes called. A detailed description of dark matter goes hand in hand with the experimental results which describe it. As a result, we will differ a detailed description of dark matter until we address these experiments, their results, and implications for the $\Lambda$CDM paradigm later in the paper. However, for the time being it is worth briefly characterizing it. Dark matter is dynamically cold (hence the C in $\Lambda$CDM), in that it moves at speeds <{}< c. This means it will remain in close proximity to baryonic matter on galactic scales an higher, as it is unable to escape the gravitational potential of baryonic matter. It cannnot interact with light, (hence dark), or if can, it does so very weakly. Furthermore, it is dissipative less, it is unable to radiate light. And finally, It is collisionless \cite{Famaey2012}

Likewise, there have been successes in describing observations of the universe by invoking dark energy. While we will wait until further discussions of experiments for a detailed characterization of dark energy, for the time being we may briefly discuss it. We may take $\Omega_{\Lambda}$ to be the dark energy component of universe's mass-energy budget. This corresponds to an energy which pervades the universe and is responsible for the expansion of the universe  \cite{scott2002}.

\section{Modified Theories of Gravity}

The earliest attempt to explain galactic-scale dynamics by changing Newton's laws of motion instead of allowing the stress energy tensor to have dark mater/energy components was offered by Milgrom in 1983. At large distances from galactic centers (at the kilo parsec scale), the accelerations of stars are on the order of $10^{-10}ms^{-2}$. He added an ad hoc interpolating function that would connect the familiar acceleration regimes of newtonian gravity to these low order regimes via a smooth transition \cite{MilgromI1983}

$$ \mu (\frac{g}{a_{0}}) \mathbf{g} =\mathbf{ g_{N}} $$

where $a_{0}$ is a constant with units of acceleration, $\mu$ is the interpolating function, g is the true local accretion due to gravity, and $g_{N}$ is the newtonian acceleration obtained from Newton's inverse square law, $\frac{GM}{r^2}$. In order to stitch the two acceleration regimes in galaxies together, we require $\mu$ tends to 1 as g > > $a_{0}$, and $\mu(x)$ tends to  $x$ for $x$ < < 1 \cite{Famaey2012}. A variety of 'families of $\mu$ functions' are discussed in the literature. However, we return to Milgrom's original discussion in 1983 to characterize it further. By choosing $\mu(x) = x(1+x^{2})^{-2}$ and $a_{0}= 2 x 10^{-8} cms^{-2}$ the rotation curves of galaxies are recovered, with no need to add dark matter or any unseen matter \cite{MilgromII1983}.

We must pause here to emphasize Milgron's model is a \emph{new model of gravity altogether}. It is a departure from Newtonian gravity, and by extension cannot be captured by Einstein gravity in the weak field limit. By inspection of the functional form, it is clearly coordinate dependent, and not lorentz invariant. While Einstein's theory can follow from first principles by extremzing the Einstein-Hilbert action, Milgrom's modified gravity by itself is motivated primarily by observed data and has no analogous theoretical underpinnings. Nonetheless, Milgrom's description has had experimental successes and continues to do so in modern times. This has prompted theoretical work to develop a parent theory of Milgrom's modified gravity.

\section{Tensor Vector Scalar Gravity}
Tensor Vector Scalar theory of gravity has been proposed as an alternative to GR, which recovers GR, Newtonian gravity, and Milgrom's gravity. Additionally, it is relativistic, and motivated by a least action principle \cite{Bekenstein2004}. As the name suggests, it contains extra degrees of freedom in its action. A metric $\tilde{g}_{\mu \nu}$ with connection $\tilde{\nabla}$, a vector field, $A_{\mu}$, and a scalar field $\phi$. They are related in the following way

$$ g_{\mu \nu} = e^{-2 \phi} \tilde{g}_{\mu \nu} - 2\sinh (2\phi) A_{\mu} A_{\nu}$$

where $ g_{\mu \nu}$ is the metic in Einstein's GR \cite{Bekenstein2006}. $A_{\mu}$ obeys a normalization condition $A_{\mu} A^{\mu} = -1$ where $ A^{\mu} = \tilde{g}^{\mu \nu} A_{\nu}$ \cite{masud2014}. The action in TeVeS has additional components, due to these degrees of freedom.

$$ S = S_{\tilde{g}} + S_{A} + S_{\phi} + S_{m} $$

S is made up of the familiar Einstein-Hilbert action, a vector field action, scalar field action, and the familiar matter action, respectively. The new actions are defined as

$$ S_{A} = -\frac{K}{32G\pi} \int [g^{\alpha \beta} g^{\mu \nu} A_{[\alpha, \mu]} A_{[\beta, \nu]} - 2(\lambda/K)(g^{\mu \nu} A_{\mu} A_{\nu} + 1 )]  (-g)^{-1/2} d^{4}x$$

$$ S_{\phi} = -\frac{1}{2} \int [\sigma^{2}h^{\alpha \beta} \phi_{,\alpha} \phi_{,\beta} + \frac{1}{2} Gl^{-2} \sigma^{4}F(kG\sigma^{2})] (-g)^{-1/2} d^{4}x$$

K is a dimensionless constant, $l$ is a scale parameter, F is an unspecified function, $\lambda$ is a lagrange multiplier filed which enforces normalization, $\sigma$ is a nondynamical scalar field, and finally $h^{\alpha \beta} = g^{\alpha \beta} - A^{\alpha} A^{\beta}$. The matter action and Einstein-Hilbert action are obtained in this description by allowing $g_{\alpha \beta} \mapsto \tilde{g_{\alpha \beta}}.$ \cite{Bekenstein2004}.

At last, we discuss the limiting cases of TeVeS. As K $\rightarrow$ 0 and $l \rightarrow \infty$ GR is recovered \cite{Bekenstein2004}. Milgrom's modified gravity can be recovered if we make the choice $F$ which satisfies

$$Kl^{2}h^{\mu \nu} \phi_{,\mu} \phi_{,\nu} = (3/4)K^{2}G^{2}\sigma^{4}(KG\sigma^{2}-2)^{2} (1 - KG\sigma^{2})^{-1} $$



\begin{thebibliography}{99}
\bibitem{Caroll2004}Carroll, Sean M. (2004), Spacetime and Geometry: An Introduction to General Relativity, San Francisco: Addison-Wesley, ISBN 0-8053-8732-3
\bibitem{scott2002} Dodelson, Scott (2008). Modern cosmology (4. [print.]. ed.). San Diego, CA : Academic Press. ISBN 978-0122191411.
\bibitem{Bertone2004} Particle Dark Matter: Evidence, Candidates, and Constraints. https://arxiv.org/abs/hep-ph/0404175
\bibitem{Famaey2012}Modified Newtonian Dynamics (MOND): Observational Phenomenology and Relativistic Extensions https://arxiv.org/abs/1112.3960
\bibitem{Plank2015} Collaboration, Planck; Ade, P. A. R.; Aghanim, N.; Arnaud, M.; Ashdown, M.; Aumont, J.; Baccigalupi, C.; Banday, A. J.; Barreiro, R. B.; Bartlett, J. G.; Bartolo, N.; Battaner, E.; Battye, R.; Benabed, K.; Benoit, A.; Benoit-Levy, A.; Bernard, J. -P.; Bersanelli, M.; Bielewicz, P.; Bonaldi, A.; Bonavera, L.; Bond, J. R.; Borrill, J.; Bouchet, F. R.; Boulanger, F.; Bucher, M.; Burigana, C.; Butler, R. C.; Calabrese, E.; et al. (2015). "Planck 2015 Results. XIII. Cosmological Parameters". arXiv:1502.01589
\bibitem{SDSS2003} K. N. Abazajian et al. "The Seventh Data Release of the Sloan Digital Sky Survey," The Astrophysical Journal Supplement. 182, 543-558 (2009).
\bibitem{MilgromI1983}  Milgrom, M., ?A modification of the Newtonian dynamics as a possible alternative to the
hidden mass hypothesis?, Astrophys. J., 270, 365?370, (1983).
\bibitem{MilgromII1983}  Milgrom, M., ?A modification of the Newtonian dynamics: Implications for galaxies?, Astrophys.
J., 270, 371?383, (1983).  http://adsabs.harvard.edu/abs/1983ApJ...270..371
\bibitem{Bekenstein2004} Bekenstein, J. D. (2004), "Relativistic gravitation theory for the modified Newtonian dynamics paradigm", Physical Review D, 70 (8)
\bibitem{Bekenstein2006}Bekenstein, J. D.; Sanders, R. H. (2006), "A Primer to Relativistic MOND Theory", EAS Publications Series, 20: 225?230
\bibitem{masud2014} Can TeVeS be a viable theory of gravity? https://arxiv.org/abs/1402.4696
\end{thebibliography}



\end{document}
